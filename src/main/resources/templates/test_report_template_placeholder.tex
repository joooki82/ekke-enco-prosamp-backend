\documentclass[a4paper,12pt]{article}

% ==============================
% Packages
% ==============================

\usepackage[utf8]{inputenc}  % UTF-8 encoding
\usepackage{graphicx}        % For including images
\usepackage{geometry}        % Adjust margins
\usepackage{fancyhdr}        % Headers and footers
\usepackage{setspace}        % Line spacing
\usepackage{xcolor}          % Colors
\usepackage{tabularx}        % Advanced tables
\usepackage{tocloft}         % For controlling Table of Contents spacing
\usepackage[magyar]{babel}   % Hungarian language support
\usepackage{titlesec}        % Control section title format
\usepackage{ragged2e}        % For text alignment
\usepackage{lmodern}         % Better font rendering
\usepackage{ulem}            % For underlining text
\usepackage{array}           % Improves table layout
\usepackage[T1]{fontenc}     % Fixes accented characters hyphenation
\usepackage{changepage}      % Adjust text width
\usepackage{makecell}        % Improved table cell formatting
\usepackage{multirow}        % Multi-row cells in tables
\usepackage{longtable}       % Multi-page tables
\usepackage{hyphenat}        % Better hyphenation
\usepackage{needspace}       % Prevent page breaks in certain places
\usepackage{lastpage}        % Reference last page in footer
\usepackage{float}           % Force figure/table placements
\usepackage{amsmath}


% ==============================
% Page Layout
% ==============================

\geometry{top=3.5cm, bottom=3.5cm, left=2.5cm, right=2.5cm}
\definecolor{aubergine}{RGB}{134,73,139} 

% Line spacing
\setstretch{1.2}

% Set default font to sans-serif
\usepackage[scaled]{helvet}  % Use Helvetica (similar to Arial)
\renewcommand{\familydefault}{\sfdefault}

%\usepackage{uarial}
%\renewcommand{\familydefault}{\sfdefault}

%\usepackage{mathpazo}  % Palatino for text & math
%\renewcommand{\familydefault}{\rmdefault}

%\usepackage{newtxtext,newtxmath}  % Times font for text & math
%\renewcommand{\familydefault}{\rmdefault}

%\usepackage{cmbright}  % Modern sans-serif font
%\renewcommand{\familydefault}{\sfdefault}

%\usepackage{fontspec}
%\setmainfont{IBM Plex Sans}  % Modern and clean





% Remove paragraph indentation
\setlength{\parindent}{0pt}

% ==============================
% Header and Footer Configuration
% ==============================

\pagestyle{fancy}
\fancyhf{}  % Clear default headers and footers
\renewcommand{\headrulewidth}{0pt}
\renewcommand{\footrulewidth}{0pt}

% First page style (with logo)
\fancypagestyle{firstpage}{
	\fancyhf{}
	\rhead{\includegraphics[width=7cm]{logo.png}}  % Logo only on first page
	\fancyfoot[C]{
		\renewcommand{\arraystretch}{0.5}
		\scriptsize
		\noindent\makebox[\textwidth]{
			\begin{tabular}{p{6.5cm} p{3cm} p{3cm} p{3.5cm}}
				\textbf{\textcolor{aubergine}{ENCOTECH Környezetvédelmi}} & \textbf{1089 BUDAPEST} & \textbf{Tel.:} +36 1 3037848 &  \texttt{meres@encotech.hu} \\
				\textbf{\textcolor{aubergine}{Szolgáltató és Tanácsadó Kft. LABORATÓRIUMA}} & Bláthy Ottó utca 41. & \textbf{Fax:} +36 1 3231512 & \texttt{www.encotech.hu} \\
			\end{tabular}
		}
	}
}

% Standard header/footer from page 2 onward
\fancypagestyle{report}{
	\fancyhf{}
	\fancyhead[L]{\footnotesize {{companyName}}, {{city}}}
	\fancyhead[R]{\footnotesize \parbox{6cm}{\raggedleft {{aimOfTest}} }}
	\renewcommand{\headrulewidth}{0.4pt}
	\renewcommand{\footrulewidth}{0.4pt}
	\fancyfoot[L]{\footnotesize Vizsgálati Jegyzőkönyv száma: {{reportNumber}}}
	\fancyfoot[C]{\footnotesize \thepage/\pageref{LastPage}}
	\fancyfoot[R]{\footnotesize ENCOTECH Kft. Laboratóriuma}
}

% ==============================
% Section Title Formatting
% ==============================

\titleformat{\section}
{\large\bfseries} % Large & Bold
{\thesection.}{1em} % Section number + spacing
{\MakeUppercase} % Uppercase section title

% ==============================
% Document Start
% ==============================
\begin{document}
	
	
% ==============================
% Fedlap
% ==============================
	% --- First Page ---
	\thispagestyle{firstpage}
	
	\begin{flushleft}
		{\footnotesize  
			\textbf{
				\makebox[6cm][s]{A NAH által NAH-1-1201/2019} \\
				\vspace{-2mm} 
				\makebox[6cm][s]{számon akkreditált vizsgálólaboratórium.}
			}
		}
	\end{flushleft}
	
	\vfill
	
	\begin{center}
		{\Huge \textbf{Vizsgálati Jegyzőkönyv}}
	\end{center}
	
	\vfill
	
	\begin{center}
		\large
		\textbf{a} \\
		\textbf{{{title}}}
	\end{center}
	
	\vfill
	
	\begin{center}
		\textbf{Témaszám:} {{projectNumber}}
	\end{center}    
	
	\vfill
	
	\begin{center}
		\textbf{A Vizsgálati Jegyzőkönyv száma:} {{reportNumber}}.
	\end{center}
	
	\begin{center}
		\textbf{A Vizsgálati Jegyzőkönyvet jóváhagyta:}
	\end{center}    
	
	\vfill
	
	\begin{center}
		\textbf{{{approvedBy}}}\\
%		{{approvedByRole}} \\
		\textit{Műszaki igazgató} \\
	\end{center}    
	
	\vfill
	
	\begin{center}
		{{issueDate}}
	\end{center}    
	
	\vfill
	
	\begin{center}
		\textbf{A Vizsgálati Jegyzőkönyv \pageref{LastPage} db számozott oldalt tartalmaz.}
	\end{center}    
	
	\begin{center}
		{\scriptsize 
			\textit{
				Az ENCOTECH Kft. Laboratóriuma írásbeli engedélye nélkül a Vizsgálati Jegyzőkönyv csak teljes terjedelmében sokszorosítható. 
				Jelen Vizsgálati Jegyzőkönyvben meghatározott eredmények csak a közölt mérési időszakokra vonatkoznak.
		}}  
	\end{center}
	
	% --- End of First Page ---
	\newpage
	
% ==============================
% Tartalomjegyzék
% ==============================

	% --- Second Page: Table of Contents ---
	\tableofcontents
	
% ==============================
% Vizsgálati jegyzőkönyv
% ==============================

	\newpage  % Move to third page
	%	--- Apply default style for all next pages ---
	\pagestyle{report}
	
	
% ==============================
% A VIZSGÁLAT CÉLJA
% ==============================
	% --- Third Page: Start of Sections ---
	\section{A VIZSGÁLAT CÉLJA}
		\begin{adjustwidth}{1cm}{0cm}
		{{aimOfTest}}
		\end{adjustwidth}
	
	
% ==============================
% A VIZSGÁLATOT VÉGEZTE
% ==============================
	\section{A VIZSGÁLATOT VÉGEZTE}
		\begin{adjustwidth}{1cm}{0cm}
			\noindent
			\textbf{ENCOTECH Környezetvédelmi Szolgáltató és Tanácsadó Kft.} \\
			1089 Budapest, Bláthy Ottó u. 41.
			
			\vspace{1.0em} % Small space for better readability
			
			% Create a tabular structure to align names properly
			\noindent
			\begin{tabular}{ p{5.5cm} p{8cm} } 
				\textit{\underline{A mintavételt végezte:}} {{samplers}}
			\end{tabular}
		\end{adjustwidth}
	
	
% ==============================
% A MEGBÍZÓ ADATAI
% ==============================	
	\section{A MEGBÍZÓ ADATAI}
		\begin{adjustwidth}{1cm}{0cm}
			
			\vspace{0.5em} % Small spacing before the table
			
			\noindent
			\renewcommand{\arraystretch}{1.4} % Adjust row height for better readability
			
			\begin{tabularx}{\textwidth}{ | p{4.5cm} | X | } % X column auto-adjusts
				\hline
				\textbf{A megbízó neve:} & \textbf{{{clientName}}} \\ \hline
				\textbf{A megbízó címe:} & {{clientAddress}} \\ \hline
			\end{tabularx}
			
		\end{adjustwidth}
		
% ==============================
% A TELEPHELY ADATAI
% ==============================
	\section{A TELEPHELY ADATAI}
		\begin{adjustwidth}{1cm}{0cm}
			\vspace{0.5em} % Small spacing before the table
			
			\noindent
			\renewcommand{\arraystretch}{1.4} % Adjust row height for better readability
			
			
			\begin{tabularx}{\textwidth}{ | p{4.5cm} | X | } % X column auto-adjusts
				\hline
				\textbf{A telephely neve:} & \textbf{{{locationName}}} \\ \hline
				\textbf{A telephely címe:} & {{locationAddress}} \\ \hline
			\end{tabularx}
			
			\vspace{0.5em} % Small spacing before the table
			
			\begin{tabular}{ p{5.5cm} p{8cm} } 
				\textit{\underline{A telephely kapcsolattartója:}} {{clientContact}}
			\end{tabular}
			
		\end{adjustwidth}
	
% ==============================
% MINTAVÉTELI ÉS ÜZEMVITELI KÖRÜLMÉNYEK
% ==============================

	\newpage  % Move to third page
	\section{MINTAVÉTELI ÉS ÜZEMVITELI KÖRÜLMÉNYEK}
	\subsection{MÉRÉSI IDŐPONT}
		\begin{adjustwidth}{1cm}{0cm}
			\begin{tabular}{ p{5.5cm} p{4cm} p{5cm}}
				\textit{\underline{Helyszíni mérések:}} {{samplingSchedule}}
			\end{tabular}
			
		\end{adjustwidth}
	
	
	
% ==============================
% KÖRNYEZETI PARAMÉTEREK
% ==============================

	\subsection{KÖRNYEZETI PARAMÉTEREK}
		\begin{adjustwidth}{1cm}{0cm}
			A mérési időszakra vonatkozó környezeti paraméterek a következők voltak. \\
			\begin{table}[h]
				\centering
				\renewcommand{\arraystretch}{1.5}
				\begin{tabular}{|c|c|c|c|}
					\hline
					\textbf{Dátum} & \multicolumn{1}{c|}{\textbf{Hőmérséklet}} & \multicolumn{1}{c|}{\textbf{Páratartalom}} & \multicolumn{1}{c|}{\textbf{Légnyomás}} \\
					& \textbf{[°C]} & \textbf{[\%]} & \textbf{[mbar]} \\
					\hline
					{{samplingDate}} & {{temperature}} & {{humidity}} & {{pressure}} \\
					\hline
				\end{tabular}\label{tab:table}
			\end{table}
			
			
		\end{adjustwidth}
	
% ==============================
% SZENNYEZŐ TECHNOLÓGIA
% ==============================
	\subsection{SZENNYEZŐ TECHNOLÓGIA}
		\begin{adjustwidth}{1cm}{0cm}
			{{technology}}
		\end{adjustwidth}
	
% ==============================
% SZENNYEZŐ TECHNOLÓGIA
% ==============================
	\subsection{MINTAVÉTELI KÖRÜLMÉNYEK, IDŐPONTOK}
	\begin{adjustwidth}{1cm}{0cm}
		{{samplingConditions}}
	\end{adjustwidth}
	\newpage  % Move to third page
	\begin{center}
		\textit{\underline{Átlagkoncentráció mintavétele}} % Underlined title
		\begin{longtable}{|m{3.5cm}|m{2cm}|m{3.5cm}|m{2cm}|m{1cm}|m{1cm}|}
			
			\hline
			\makecell{\textbf{Minta jele /} \\ \textit{Vizsgált szennyező}} & \makecell{Mintavétel \\ ideje} &\makecell{ Munkaterület} & \makecell{Mintavétel \\ jellege} & \makecell{Hőm. \\ °C} & \makecell{Párat. \\ \%} \\
			\hline
			
			\endfirsthead
			
			\hline
			\makecell{\textbf{Minta jele /} \\ \textit{Vizsgált szennyező}} & \makecell{Mintavétel \\ ideje} &\makecell{ Munkaterület} & \makecell{Mintavétel \\ jellege} & \makecell{Hőm. \\ °C} & \makecell{Párat. \\ \%} \\
			\hline
			\endhead
			
			\hline
			%			\multicolumn{6}{|r|}{{a táblázat a következő oldalon folytatódik}} \\ \hline
			\endfoot
			
			\hline
			\endlastfoot

            {{sampleDetailsAverage}}
			
		\end{longtable}
	\end{center}
	
	\newpage  
	\begin{center}
		\textit{\underline{Csúcskoncentráció mintavétele}} % Underlined title
		\begin{longtable}{|m{3,5cm}|m{2cm}|m{3,5cm}|m{2cm}|m{1cm}|m{1cm}|}
			
			\hline
			\makecell{\textbf{Minta jele /} \\ \textit{Vizsgált szennyező}} & \makecell{Mintavétel \\ ideje} &\makecell{ Munkaterület} & \makecell{Mintavétel \\ jellege} & \makecell{Hőm. \\ °C} & \makecell{Párat. \\ \%} \\
			\hline
			
			\endfirsthead
			
			\hline
			\makecell{\textbf{Minta jele /} \\ \textit{Vizsgált szennyező}} & \makecell{Mintavétel \\ ideje} &\makecell{ Munkaterület} & \makecell{Mintavétel \\ jellege} & \makecell{Hőm. \\ °C} & \makecell{Párat. \\ \%} \\
			\hline
			\endhead
			
			\hline
			%			\multicolumn{6}{|r|}{{a táblázat a következő oldalon folytatódik}} \\ \hline
			\endfoot
			
			\hline
			\endlastfoot



            {{sampleDetailsPeak}}

			
		\end{longtable}
	\end{center}
	
	
% ==============================
% FELHASZNÁLT ESZKÖZÖK, MINTAVÉTELI ÉS MÉRÉSI MÓDSZEREK
% ==============================
	\newpage
	\section{FELHASZNÁLT ESZKÖZÖK, MINTAVÉTELI ÉS MÉRÉSI MÓDSZEREK}
	\subsection{A KÖRNYEZETI LEVEGŐ ÁLLAPOTJELZŐINEK MEGHATÁROZÁSA}

		{{equipmentList}}

	\subsection{SZENNYEZŐANYAG KONCENTRÁCIÓ MEGHATÁROZÁSA}
	
	\begin{adjustwidth}{1cm}{0cm}

		\vspace{1.0em} % Small space for better readability

		{{determinationOfPollutantConcentration}}

		\vspace{1.0em}

	\end{adjustwidth}
	
	
% ==============================
% A VIZSGÁLAT SORÁN ALKALMAZOTT SZABVÁNYOK, ELJÁRÁSOK
% ==============================
	\newpage
	\section{A VIZSGÁLAT SORÁN ALKALMAZOTT SZABVÁNYOK, ELJÁRÁSOK}
	
		\renewcommand{\arraystretch}{1.1} % Adjust row height for better readability
		\footnotesize % Decrease font size by one scale

		\begin{longtable}{|p{5cm}|p{10cm}|}
			\hline
			\multicolumn{2}{|c|}{\textbf{Mintavétel, helyszíni vizsgálatok}} \\
			\hline
			\endfirsthead

			\multicolumn{2}{|c|}{\textbf{Mintavétel, helyszíni vizsgálatok}} \\
			\hline
			\endhead

			\endfoot

			\hline
			\endlastfoot

			{{samplingStandards}}

		\end{longtable}

		\begin{longtable}{|p{5cm}|p{10cm}|}
			\hline
			\multicolumn{2}{|c|}{\textbf{Szennyezőanyag-tartalom meghatározása - ENCOTECH Kft.}} \\
			\hline
			\endfirsthead

			\hline
			\multicolumn{2}{|c|}{\textbf{Szennyezőanyag-tartalom meghatározása - ENCOTECH Kft.}} \\
			\hline
			\endhead

			\endfoot

			\hline
			\endlastfoot

			{{encotechStandards}}

		\end{longtable}

		\begin{longtable}{|p{5cm}|p{10cm}|}
			\hline
			\multicolumn{2}{|c|}{\textbf{Szennyezőanyag-tartalom meghatározása - BÁLINT ANALITIKA Kft.}} \\
			\hline
			\endfirsthead

			\hline
			\multicolumn{2}{|c|}{\textbf{Szennyezőanyag-tartalom meghatározása - BÁLINT ANALITIKA Kft.}} \\
			\hline
			\endhead

			\endfoot

			\hline
			\endlastfoot

			{{balintStandards}}

		\end{longtable}


		\parbox{\textwidth}{\raggedright \footnotesize
			* Magyar Szabványügyi Testület által visszavont szabvány, amelyet a Nemzeti Akkreditáló Hatóság továbbra is alkalmazható módszernek tekint.
		}


% ==============================
% VIZSGÁLATI EREDMÉNYEK
% ==============================
	\newpage
	\section{VIZSGÁLATI EREDMÉNYEK}
		\begin{adjustwidth}{1cm}{0cm}
			\textit{A mérési eredmények a munkahelyi légtérnek a vizsgálat ideje alatt érvényes jellemzőire vonatkoznak.}
		\end{adjustwidth}

	\subsection{ÁTLAGKONCENTRÁCIÓ MINTAVÉTELE}
	
		\begin{center}
			\begin{longtable}{|m{2.5cm}|m{5cm}|m{3cm}|m{2cm}|m{2cm}|}

				\hline
				\makecell{\textbf{Minta jele}} & \makecell{Szennyezőanyag} &\makecell{Leválasztott \\ mennyiség \\ $[\mathrm{\mu g}]$} & \makecell{Minta  \\ térfogat \\ $[\text{m}^3]$\\ } & \makecell{Koncentráció \\ $[\text{mg/m}^3]$} \\
				\hline

				\endfirsthead

				\hline
				\makecell{\textbf{Minta jele}} & \makecell{Szennyezőanyag} &\makecell{Leválasztott \\ mennyiség \\ $[\mathrm{\mu g}]$} & \makecell{Minta  \\ térfogat \\ $[\text{m}^3]$\\ } & \makecell{Koncentráció \\ $[\text{mg/m}^3]$} \\
				\hline
				\endhead

				\hline
				%			\multicolumn{6}{|r|}{{a táblázat a következő oldalon folytatódik}} \\ \hline
				\endfoot
				
				\hline
				\endlastfoot

				{{averageConcentration}}

				\vspace{-\baselineskip}
				\parbox{\textwidth}{\raggedright \footnotesize
				*20 °C-ra és 101,3 kPa légköri nyomásra számított koncentráció
				}
			\end{longtable}

			
			
		\end{center}

		\newpage

		\subsection{CSÚCSKONCENTRÁCIÓ MINTAVÉTELE}

		\begin{center}
			\begin{longtable}{|m{2.5cm}|m{5cm}|m{3cm}|m{2cm}|m{2cm}|}
				
				\hline
				\makecell{\textbf{Minta jele}} & \makecell{Szennyezőanyag} &\makecell{Leválasztott \\ mennyiség \\ $[\mathrm{\mu g}]$} & \makecell{Minta  \\ térfogat \\ $[\text{m}^3]$\\ } & \makecell{Koncentráció \\ $[\text{mg/m}^3]$} \\
				\hline

				\endfirsthead

				\hline
				\makecell{\textbf{Minta jele}} & \makecell{Szennyezőanyag} &\makecell{Leválasztott \\ mennyiség \\ $[\mathrm{\mu g}]$} & \makecell{Minta  \\ térfogat \\ $[\text{m}^3]$\\ } & \makecell{Koncentráció \\ $[\text{mg/m}^3]$} \\
				\hline
				\endhead
				
				\hline
				%			\multicolumn{6}{|r|}{{a táblázat a következő oldalon folytatódik}} \\ \hline
				\endfoot
				
				\hline
				\endlastfoot

				{{peakConcentration}}

				\vspace{-\baselineskip}
				\parbox{\textwidth}{\raggedright \footnotesize
				*20 °C-ra és 101,3 kPa légköri nyomásra számított koncentráció
				}

				\parbox{\textwidth}{\raggedright \footnotesize
				Budapest, {{issueDate}}				}
%				\noindent\textbf{Budapest, {{issueDate}}} % Date aligned to the left

				\vspace{1em} % Adjust vertical space

				\begin{center}
					\begin{tabular}{p{7cm} c m{7cm}} % Column widths and alignment
						\textbf{A Vizsgálati Jegyzőkönyvet készítette:} & &
						\begin{center}
							\textbf{{{preparedBy}}} \\
							\textbf{Vizsgáló mérnök} \\
%							{{preparedByRole}}
						\end{center} \\[6em] % Increased space for signature

						\textbf{A Vizsgálati Jegyzőkönyvet ellenőrizte:} & &
						\begin{center}
							\textbf{{{checkedBy}}} \\
							\textbf{Vizsgáló mérnök} \\
%							{{checkedByRole}}
						\end{center} \\
					\end{tabular}
				\end{center}

				\vspace{1em}

				\begin{center}
					\textbf{-- Vizsgálati Jegyzőkönyv vége --}
				\end{center}


			\end{longtable}



		\end{center}






	% --- Appendices Section (MELLÉKLETEK) ---
	\newpage
	\appendix
	\section{MELLÉKLETEK}  % This will appear in the Table of Contents
	
	\subsection{1. Melléklet - Mérési adatok}  
	A mérési adatok részletezése.
	
	\subsection{2. Melléklet - Egyéb adatok}  
	Egyéb adatok részletezése.
	
	
\end{document}
